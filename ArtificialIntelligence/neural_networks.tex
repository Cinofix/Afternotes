\section{Neural Networks}
A Neural Network is an information processing paradigm that is inspired by the way biological nervous systems, such as the brain, process information. The key element of this paradigm is the new structure of the information processing system. It is composed of a large number of highly interconnected processing elements (\textbf{neurons}) working in unison to solve specific problems. A NN is configured for a specific application, such as pattern recognition or data classification, through a learning process.\\
Two key features distinguish neural networks from any other sort of computing developed:
\begin{itemize}
	\item \textbf{Neural networks are adaptive or trainable}. Neural networks are not so much programmed as they are trained with data, the more data they are fed, the more accurate or complete is their response.
	\item \textbf{Neural network are naturally massively parallel}. This suggests they should be able to make decisions at high-speed and be fault tolerant (information is stored in a distributed fashion). There are more interconnections than processing units and the processing power of a neural network corresponds to the number of interconnection updates per second. 
\end{itemize}

\paragraph*{The Neuron.} The biological neural network is composed by the following elements:
\begin{itemize}
	\item \textbf{Cell body (Soma)}: 5-10 microns in diameter, it is the \textit{computational unit}.
	\item \textbf{Axon}: Output mechanism for a neuron.
	\item \textbf{Dendrites}: Receive incoming signals from other nerve axons via synapse.
\end{itemize}

\paragraph*{The neural dynamics.} The transmission of signal in the cerebral cortex is a complex process:
$$\text{Electrical} \rightarrow \text{Chemical} \rightarrow \text{Electrical}$$
Simplifying:
\begin{enumerate}
	\item The cellular body performs a "\textit{weighted sum}" of the incoming signals.
	\item If the result exceeds a certain threshold value, then it produces an "action potential" which is sent down the axon (cell has "fired"), otherwise it remains in a rest state.
	\item When the electrical signal reaches the synapse, it allows the “neuro-transmitter” to be released and these combine with the “receptors” in the post-synaptic membrane.
	\item The post-synaptic receptors provoke the diffusion of an electrical signal in the post-synaptic neuron. 
\end{enumerate}
In other words computations by neuron are performed thanks to a combination of electrical and chemical processes. 
\image{img/neural_dynamics}{Neural dynamics.}{0.6}

\paragraph*{Synaptic Efficacy.} Synapses are connecting point between neurons. It’s the amount of electricity that enters into the post-synaptic neuron, compared to the action potential of the pre-synaptic neuron. The \textbf{learning step} takes place by modifying the synaptic efficacy. Two different types of synapses are present:
\begin{itemize}
	\item \textbf{Excitatory}: favor the generation of action potential in the post-synaptic neuron.
	\item \textbf{Inhibitory}: hinder the generation of action potential.
\end{itemize}

\subsection{The McCulloch and Pitts Model}
Neural network simulations appear to be a recent development, however, this field was established before the advent of computers and the first model was created in the 1943, it is called \textbf{McCulloch and Pitts Model}.\\
The McCulloch-Pitts (MP) neuron is a simple process unit modeled as a binary threshold unit.
\image{img/MP_neuron}{McCulloch and Pitts neuron representation.}{0.55}
The unit fires if the \textbf{net input} $\sum_j w_jI_j$ reaches (or exceeds) the unit's threshold $T$:
$$y = g\left(\sum_j w_jI_j - T\right) \qquad \text{where } g \text{ is the unit step function}$$
If neuron is firing, then its output $y$ is 1, otherwise it is 0.
$$g \text{ is the unit step function such that} \quad \rightarrow \quad g(x) = 
\begin{cases}
0 \quad \text{if }x<0\\
1 \quad \text{if }x \geq 0
\end{cases}$$
The weight $w_{ij}$ represents the strength of the synapse between neuron $j$ and neuron $i$. The function $g(x)$ is also called \textbf{activation function}.

\paragraph*{Properties of MP networks.} Combining the MP neurons, it is possible to simulate the behavior of any Boolean circuit.
\image{img/MP_properties}{These elementary logical operations (\textit{a}) \textbf{negation}, (\textit{b}) \textbf{and}, (\textit{c}) \textbf{or}. In each diagram the states of the neurons on the left are at time $t$ and those on the right at time $t+1$.}{0.65}

\subsection{Network Topologies and Architectures} There are different network topologies and the main differences are highlighted below.

\begin{table}[H]
	\centering
	\begin{tabular}{| p{7.5cm} | p{7.5cm} |}
		\hline
		\textbf{Feedforward only} allow signals to travel one way only: from input to output. There are no feedback (loops). The output of any layer does not affect that same layer. This type of organization is also referred to as bottom-up or top-down. & \textbf{Feedback networks} (\textit{or Recurrent networks}) can have signals traveling in both directions by introducing loops in the network. Feedback networks are powerful and can get extremely complicated.\\
		\hline
		\textbf{Fully connected} has each neuron connected to every neuron in the previous layer, and each connection has it's own weight. & \textbf{Sparsely connected} has fewer links than the possible maximum number of links within that network.  \\
		\hline
		\textbf{Single layer} & \textbf{Multilayer}\\
		\hline
	\end{tabular}
\end{table} 
\image{img/feedforward}{\textit{(a)} feedforward network - \textit{(b)} feedback network}{0.65}
\subsection{Classification problems}
Given:
\begin{enumerate}
	\item \textbf{Features}, attributes that describe our objects. $f_1,f_2,\dots,f_n$
	\item \textbf{Classes}, categories in which the objects are divided. $c_1,\dots,c_m$
\end{enumerate}
What we desire is to classify our objects according to its features. A neural networks can be used as a classification device and it can be configured as follow:
\begin{itemize}
	\item The \textit{input} of the network are the object's features to classify.
	\item The \textit{output}, returned by the network, is the class predicted for the input object.
\end{itemize}
Before going on we assume that features are numbers, and remember that we cannot mapping categorical features into numbers, since otherwise we can introduce an order which can make confusion or error. Example: red = 0, blu = 1, green = 2, is not a correct coding since we are imposing an incorrect order between colors. We propose a simple model of neural network:

\image{img/NN_1}{Example of NN with 3 features input and 2 class labels as output.}{0.4}
The application of a learning algorithm of a network configuration consists on finding the best configuration of weights of the income connection and the threshold.
In these classification problems we can get rid of the \textbf{thresholds} (also called bias) associated to neurons by adding an extra unit \textbf{permanently clamped at -1}. Doing this, thresholds become weights and can be adaptively adjusted during learning phase, otherwise we have to tune manually the right threshold. Since it is just a simple parameter we can learn it taking advantage on the learning procedure. 
$$NET_i = \sum w_{ij} x_j -1w_0$$
\image{img/NN_thresholds}{Thresholds}{0.4}

\subsection{The Perceptron}
The perceptron is an algorithm for \textbf{supervised learning} of binary classifiers (functions that can decide whether an input, represented by a vector of numbers, belongs to some specific class or not). It is a type of linear classifier, that is a classification algorithm that makes its predictions based on a linear predictor function combining a set of weights with the feature vector. A simple perceptron can only solve linearly separable problems, this make the model very limited in terms of computational power. However, it is possible to say that for the problems that it can solve, it is able to learn from example and it can be considered as the \textbf{simplest feedforward neural network}. It is implemented as a network of one layer of M\&P neurons connected in a feedforward way.

\paragraph*{The Perceptron Learning Algorithm.} It is an iterative algoritnm characterized by the following variables and parameters:
\begin{itemize}
	\item Inputs vector: $x(n) = (m+1)\text{-by-1 input vector} = [-1, x_1(n), x_2(n), \dots, x_m(n)]^T$\\
	Objects are described with $m$ features and -1 as the  threshold, that is the reason of the $m+1$.
	\item Weights vector: $w(n) = (m+1)\text{-by-1 weight vector} = [b, w_1(n), w_2(n), \dots, w_m(n)]^T$
	\item $b$ = bias
	\item $y(n)$ = actual response (quantized). $\quad +1/-1$
	\item $d(n)$ = desired response. $\quad +1/-1$
	\item $\eta$ = learning-rate parameter, a positive constant less than unity. It affects the convergence of the learning algorithm. 
\end{itemize}
It is structured with the following steps:
\begin{enumerate}
	\item \textbf{Initialization.} Set $w(0) = 0$ and then perform the following computations for time-step $n=1,2,\dots$.
	\item \textbf{Activation.} At time-step $n$, activate the perceptron by applying continuous-valued input vector $x(n)$ and desired response $d(n)$. Pick element from the training set.
	\item \textbf{Computation of Actual Response.} Compute the actual response of the perceptron as:
	$$y(n) = sgn[w^T(n)x(n)] \qquad w^Tx = \sum_iw_ix_i$$
	where $sgn(\cdot)$ is the signum function. Until this phase there's not learning.
	\item \textbf{Adaptation of Weight Vector.} Update the weight vector of the perceptron to obtain.
	$$w(n+1) = w(n) + \eta[d(n) - y(n)]x(n)$$
	where
	$$d(n) = 
	\begin{cases}
	+1 \qquad \text{if } x(n) \text{ belongs to class } \varphi_1\\
	-1 \qquad \text{if } x(n) \text{ belongs to class } \varphi_2 
	\end{cases}$$
	In this point there's the learning phase. In case of wrong classification, the current configuration will change modifying $d(n) - y(n)$. The convergence of the algorithm is ensured by $x(n)$.If the network classify correctly then there is no learning $d(n)-y(n) = 0$. Note that $x(n)$ is included to be sure that the new prediction is correct but it can fail with the previous one.
	\item \textbf{Continuation.} Increment time step $n$ by one and go back to step 2.
\end{enumerate}
Through this algorithm we are looking for a straight line which separates the good examples from bad ones. The decision boundary is described such that $w_1x_1 + w_2x_2 = 0$ and in a general way each point is classified according the following formula:
$$\text{net}_i = \sum_i w_ix_i \rightarrow \begin{cases}
\text{if } >0 \rightarrow +1\\
\text{if } <0 \rightarrow -1
\end{cases}$$
\image{img/perceptron_algorithm}{Perceptron learning algorithm procedure.}{0.65}
From a geometrical point of view, finding $w$ means find the line (or hyperplane) which separates the two regions.
As it is possible to see from the previous image, some examples can be misclassified. Each time a precise error is corrected but other mistakes can be produced, continuing to correct the errors found, the final situation can be reached.\\
It is possible to define a \textbf{decision region} as an area wherein all examples of ones class fall. A classification problem is said to \textbf{linearly separable} if the decision regions can be separated by an hyperplane. These concepts allows us to introduce the fact that one important \textbf{limitation of perceptrons} is that they can only solve linearly separable problems.

\paragraph*{The Perceptron Convergence Theorem.} This theorem was formulated by Rosenblatt in 1960. It said that:\\
If the training set is \textbf{linearly separable}, the perceptron learning algorithm \textbf{always converges} to a consistent hypothesis after a \textbf{finite} number of epochs, for any $\eta > 0$ (It doesn't say anything about the number of steps).\\
If the training set is \textbf{not linearly separable}, after a certain number of epochs the weights start oscillating. In this situation surely some points will be misclassified. The learning algorithm will never converge to a stable configuration, due to the fact that perceptron can't solve non-linear classification problem.

\par \bigskip \noindent
\subsection{Multi-Layer Feedforward Networks}
In the following image it is possible to find some properties of the different structures that can create a neural network.
\image{img/role_units}{A view of the Role of Units}{0.86}
A single layer represents the structure of a perceptron and as we said before it can implement only linearly separable functions. In order to overcome these limitations, the introduction of multi-layer feedforward networks allows us to improve our networks through the addition of \textbf{hidden layers} between the input and the output layer. This can allow us to reach a good approximation of our classification problem, it is possible to notice that a network with just one hidden layer can represent any Boolean functions including XOR. From the \textbf{Universal approximation power} we also know that a two-layer network (only one hidden layer) can approximate any smooth function (valid for regression problem). But this power does not give us information about how large should be this network. This model here is static, meaning that we have no feedback.
$$f: I \rightarrow \mathbb{ R }^m \qquad I \subseteq \mathbb{ R }^n$$
$$f(x_1,\dots,x_n) = \bar{y}\in \mathbb{ R }^m$$
We want that the estimated function $NN(X)$ satisfies the following property:
$$||\text{NN}(\bar X)- f(\bar X)|| \leq \varepsilon$$
\image{img/fNN.png}{Neural Network function model.}{0.65}		\image{img/data-bias-variance.png}{Neural Network function approximation.}{0.65}
Note that the Universal approximation power works only for regression problems, since usually classification problems are more difficult (3 layers instead of 2).

\paragraph*{Continuous-Valued Units.} The usage of continuous-valued units allows us to introduce calculus and derivatives procedures that will give us several advantages.\\ 
\textbf{Sigmoid} (or logistic), this is the rate of firing of the neurons, they can fire more or less, so they are not Boolean 0 or 1 but instead they are float values from 0 to 1. They are often interpreted as probabilities.
$$g \rightarrow \sigma(x) = \frac{1}{1+e^{-f(x)}} \in (0,1)$$
\image{img/sigmoid}{Sigmoid function.}{0.51}
Continuous-Valued discriminant functions allows us to have a confidence about the prediction. We will see later on that they are used for finding optimal solutions using the gradient descent technique. In particular, when the $g(z)$ is close to zero (less confidence about the prediction) the gradient will be greater then the situation in which $|g(x)| >> 0$ (more confidence, weights could remain stable).\\

\textbf{Hyperbolic tangent} is a rescaling of the logistic sigmoid such that its output range is from -1 and 1. This type of function could be better because the length of the interval has length 2 and so it has more degrees of freedom.
$$g \rightarrow \sigma(x) = \frac{e^{f(x)} - e^{-f(x)}}{e^{f(x)} + e^ {-f(x)}} \in (- 1,1)$$
\image{img/hyperbolic}{Hyperbolic tangent function.}{0.45}

\subsection{Back-propagation Learning Algorithm}
It is an algorithm for learning the weights in a feed-forward network, solving the learning in multilayers neural networks (respect to the one provided by the perceptron). We've got a training set $\mathcal{L} = \{(x_1, y_1), \dots, (x_n,y_n) \}$ in which each pair is an input/output behavior. The algorithm is based on gradient descent method, it can be considered as a greedy algorithm with infinite possible solutions where each time the best solution is chosen. The algorithm is guarantee to converge only to a local maximum and during the execution we move along the gradient through a predetermined step-size. This property is due to the fact that this algorithm is just an approximation based on a greedy solution.
\image{img/backpropagation}{Back-propagation schema. $W^l_{ij}$ represents the weight on connection between the $i^{th}$ unit in layer $(l-1)$ to $j^{th}$ unit in layer $l$.}{0.70}

\paragraph*{Supervised Learning.} Supervised learning algorithms require the presence of a previous knowledge that makes possible to provides the right answers to the input questions. Technically this means that we need a \textbf{training set} of the form:
$$\mathcal{L} =\left\{\left(x^1, y^1 \right), \dots , \left(x^p, y ^p \right) \right\}$$
where:
$$x^\mu (\mu=1,\dots, p) \text{ is the network input vector}$$
$$y^\mu (\mu=1,\dots, p) \text{ is the desired network output vector}$$
The learning (or training) phase consists of determining a configuration of weights in such a way that the network output be as close as possible to the desired output, for all the examples in the training set. Normally, this amounts to minimizing an \textbf{error function} such as:
$$E(w) = \frac{1}{2} \sum_{\mu} \sum_{k} \left(y_k^\mu - O_k^\mu(w)\right)^2$$
where $O_k^\mu(w)$ is the output provided by the output unit $k$ when the network is given example $\mu$ as input. In other words, $O_k^\mu(w)$ represents the prediction of the $k^{th}$ unit when the input is $\mu$. The loss function computes the difference between the network output and the expected output and of course, our aim is to \textbf{minimize the loss function} $min_w E(w)$ that means finding the set of weights that minimize the function. It is possible to say so that we are in a minimization (optimization) problem. If the network is performing well $E(\omega)$ is close to zero.\\
In order to minimize the error function $E$, we can use the classic \textbf{gradient-descent} algorithm. Gradient gives information about the best path to follow in order to increase/decrease our objective function. Without any kind of direction information it would be extremely complex to find the solution since we have to search along an infinite number of directions in a continuous domain. Gradient descent is a very well-known greedy algorithm, but it does not guarantee to find an optimal solution, but at least a local one. It is a first-order iterative optimization algorithm for finding the minimum of a function. It can be defined as:
$$w_{ji} \leftarrow w_{ji} - \eta \frac{\partial E_k}{\partial w_{ji}} \qquad \text{where } \eta \text{ is the learning rate.}$$
In back-propagation algorithm, we have an initial configuration of the network and at each step we update the weights as:
$$w_{new} = w_{old} - \eta \nabla E(w_{old})$$
To compute the partial derivatives we use the \textbf{error back propagation} algorithm, it consists of two stages:
\begin{itemize}
	\item \textbf{Forward pass}: the input of the network is propagated layer after layer in forward direction.
	\item \textbf{Backward pass}: the "error" made by the network is propagated backward and weights are updated properly.
\end{itemize}
In a more informal way it is possible to say that we determine the gradient direction and then we will move in the opposite direction since we want to minimize the error.

\paragraph*{Notations.} Before start understanding in which way the weights are updated, it is important to fix some notations that will be used later.\\
Given a pattern $\mu$, hidden unit $j$ receives a net input
$$h _ { j } ^ { \mu } = \sum _ { k } w _ { j k } x _ { k } ^ { \mu }$$
and produces as output:
$$V _ { j } ^ { \mu } = g \left( h _ { j } ^ { \mu } \right) = g \left( \sum _ { k } w _ { j k } x _ { k } ^ { \mu } \right)$$
\imageb{img/notations}{0.5}

\paragraph*{Updating Hidden-to-Output Weights.} The updating rules of the back-propagation algorithm are nothing but a long series of chain rules, for this reason, the objective function is not linear and so the output is highly non linear, indeed, it is a product of chain of non linear functions.\\
\begin{equation} \notag
\begin{split}
\Delta w_{ij} &= -\eta \frac{\partial E}{\partial w_{ij}}\\
&= -\eta \frac{\partial}{\partial w_{ij}} \left[\frac{1}{2} \sum_{\mu} \sum_k (y_k^\mu - O_k^\mu)^2\right] \quad \text{First application of the chain rule}\\
&= \eta \sum _{\mu} \sum_{k} \left(y_{k}^{\mu} - O_{k}^{\mu} \right) \frac{\partial O_{k}^{\mu}} {\partial w_{ij}} \quad \text{The sum over } k \text{ disappers because the partial derivative is}\\
& \hspace*{13.5em} \text{different from 0 only when }k=i \\
&= \eta \sum_{\mu} \left(y_{i}^{\mu} - O_{i}^{\mu} \right) \frac{\partial O_{i}^{\mu}} {\partial w_{ij}} \qquad \text{Again we apply the chain rule. Partial derivative can be}\\
& \hspace*{12.5em} \text{written as } w_{ij} \rightarrow h_i^\mu \rightarrow g'(h_i^\mu)\\
&= \eta \sum _ { \mu } \left( y _ { i } ^ { \mu } - O _ { i } ^ { \mu } \right) g ^ { \prime } \left( h _ { i } ^ { \mu } \right) V _ { j } ^ { \mu } \qquad \text{That is } \frac{\partial h_i^\mu}{\partial w_{ij}} = \frac{\partial}{\partial w_{ij}}\left(\sum_l w_{il} V_l \right) = V_j \frac{\partial w_{ij}}{\partial w_{ij}} = V_j\\
&= \eta \sum _ { \mu } \delta _ { i } ^ { \mu } V _ { j } ^ { \mu } \qquad \qquad where: \quad \delta _ { i } ^ { \mu } = \left( y _ { i } ^ { \mu } - O _ { i } ^ { \mu } \right) g ^ { \prime } \left( h _ { i } ^ { \mu } \right)
\end{split}	
\end{equation}
In $\eta \sum_{\mu} \delta_{i}^{\mu} V_{j}^{\mu}$, the component $\delta_{i}^{\mu}$ represents the error made by the $i$-th neuron and instead $V_{j}^{\mu}$ is the output of the neuron.  

\paragraph*{Updating Input-to-Hidden Weights.}
\begin{equation} \notag
\begin{split}
\Delta w _ { j k } &= - \eta \frac { \partial E } { \partial w _ { j k } } \\
&= \eta \sum _ { \mu } \sum _ { i } \left( y _ { i } ^ { \mu } - O _ { i } ^ { \mu } \right) \frac { \partial O _ { i } ^ { \mu } } { \partial w _ { j k } } \qquad \text{Chain rule}\\
&= \eta \sum _ { \mu } \sum _ { i } \left( y _ { i } ^ { \mu } - O _ { i } ^ { \mu } \right) g ^ { \prime } \left( h _ { i } ^ { \mu } \right) \frac { \partial h _ { i } ^ { \mu } } { \partial w _ { j k } } \qquad \text{Chain rule}
\end{split}
\end{equation}
We have that the partial derivative is:
\begin{equation} \notag
\begin{split}
\frac { \partial h _ { i } ^ { \mu } } { \partial w _ { j k } } &= \sum _ { l } w_{ i l } \frac { \partial V _ { l } ^ { \mu } } { \partial w_{ j k } }\\
&= w_ { i j } \frac { \partial  V_ { j } ^ { \mu } } { \partial  w _ { j k } }\\
&= w_{ i j } \frac { \partial g \left( h _ { j } ^ { \mu } \right) } { \partial w _ { j k } }\\
&= w_{ i j } g ^ { \prime } \left( h _ { j } ^ { \mu } \right) \frac { \partial h _ { j } ^ { \mu } } { \partial w _ { j k } }\\
&= w_{ i j } g ^ { \prime } \left( h _ { j } ^ { \mu } \right) \frac { \partial } { \partial w _ { j k } } \sum _ { m } w _ { j m } x _ { m } ^ { \mu }\\
&=  w_{ i j } g ^ { \prime } \left( h _ { j } ^ { \mu } \right) x _ { k } ^ { \mu }
\end{split}
\end{equation}
Hence coming back to the original equation we have that:
\begin{equation}
\begin{split}
\Delta w _ { j k } &= \eta \sum _ { \mu , i } \left( y _ { i } ^ { \mu } - O _ { i } ^ { \mu } \right) g ^ { \prime } \left( h _ { i } ^ { \mu } \right) w_ { i j } ~g ^ { \prime } \left( h _ { j } ^ { \mu } \right) x _ { k } ^ { \mu }\\
&= \eta \sum _ { \mu , i } \delta _ { i } ^ { \mu } w_ { i j } g ^ { \prime } \left( h _ { j } ^ { \mu } \right) x _ { k } ^ { \mu }\\
&= \eta \sum _ { \mu } \hat { \delta } _ { j } ^ { \mu } x _ { k } ^ { \mu } \qquad \qquad where: \quad \hat { \delta } _ { j } ^ { \mu } = g ^ { \prime } \left( h _ { j } ^ { \mu } \right) \sum _ { i } \delta _ { i } ^ { \mu } w_ { i j }
\end{split}
\end{equation}
$\sum_{i} \delta_{i}^{\mu} W_{ij}$ is the average error performed by the output layer and $i$ is the reference to the $i$-th neuron.\\
In the following image it is possible to understand the error back-propagation in which the black line means the forwarded signal instead the red lines indicate the error that is back-propagated.
\image{img/error_backpropagation}{Error Back-propagation representation.}{0.6}

\paragraph*{Locality of Back-Propagation.} Another important aspect is the locality of the back-propagation.
\image{img/locality_backpropagation}{Locality of back-propagation}{0.2}
\textbf{Off-line:} In this way we compute the gradient exactly. It is done a sum over all the possible $\mu$. Normally this procedure is not used because it is computational expensive and also because if we start search in a bad region we can't escape from it.
$$\Delta \omega_{pq} = \eta \sum_{\mu} \delta_{p}^{\mu} V_{q}^{\mu} \qquad \qquad \text{off-line}$$
$$\delta_{p}^{\mu} = \text{Error} \quad V_{q}^{\mu} = \text{Output of neurons}$$
\textbf{On-line:} In this other situation some noise is introduced. It allows to escape a bad region and to reach a better solution through the exploration of regions that otherwise we won't visit.
$$\Delta \omega_{pq} = \eta\delta_ {p}^{\mu} V_{q}^{\mu} \qquad \qquad \text{on-line}$$
A possible compromise between the two techniques is to use the first technique with a small training set, this third option is called (\textbf{stochastic gradient descent}).

%TODO: Da commentare
\paragraph*{The Back-Propagation Algorithm.} In this algorithm we will use the on-line rule.\\
The general structure of the algorithm:
\begin{enumerate}
	\item Initialize the weight to (\textit{small}) random values. Small because otherwise derivative of logistic function will be all zeros.
	%TODO: Immagine
	\item Choose a pattern $\bar{x}^\mu$ and apply it to the input layer $(m=0)$.
	$$V _ { k } ^ { 0 } = x _ { k } ^ { \mu } \qquad \forall k$$
	\item Propagate the signal forward, remember that $g$ is a continuous function:
	$$V _ {i} ^ { m } = g \left( h _ { i } ^ { m } \right) = g \left( \sum _ { j } w _ { i j } V _ { j } ^ { m - 1 } \right)$$
	\item Compute the $\delta$'s for the output layer:
	$$\delta _ { i } ^ { m } = g ^ { \prime } \left( h _ { i } ^ { m } \right) \left( y _ { i } ^ { m } - V _ { i } ^ { m } \right)$$
	\item Compute the $\delta$'s for all preceding layers (for each neuron):
	$$\delta _ { i } ^ { m - 1 } = g ^ { \prime } \left( h _ { i } ^ { m - 1 } \right) \sum _ { j } w _ { j i } ^ { m } \delta _ { j } ^ { m }$$
	\item Update connection weights according to the error, with learning rate $\eta$:
	$$w _ { i j } ^ { N E W } = w _ { i j } ^ { O L D } + \Delta w _ { i j } \qquad \text { where } \quad \Delta w _ { i j } = \eta \delta _ { i } ^ { m } V _ { j } ^ { m - 1 }$$
	\item Go back to step 2 until convergence.
\end{enumerate}
One of the most important problem is the fact that $\eta$ (learning step) should be initialized in a correct way. When $\eta$ is small the algorithm will converge but it will be too slow, on the other hand when it is too big, there will be an oscillating problem that won't bring the algorithm to the convergence. \\
In the following image four trajectories can be found, the only significant difference between them is the value of $\eta$ they used. From left to right they are $0.02, 0.0476, 0.049, 0.0505$.
\image{img/learning_rate}{The Role of the Learning Rate. The minimum is at the $+$ and the ellipse shows a constant error contour.}{0.9}
The first case is too slow in reaching the minimum, in the second and in the third case the oscillation becomes smaller and smaller, instead in the last case the algorithm won't converge, the oscillation is larger and larger.\\
As said before, one of the main limitations of the \textbf{gradient descent} method is the fact that:
\begin{itemize}
	\item It converges too slowly if $\eta$ is too small.
	\item It oscillates if $\eta$ is too large.
\end{itemize}
One possible remedy to this problem is in introducing the \textbf{momentum term}, a simple heuristic that can help to find a good value for learning rate. One of the main advantages of this improvement is the fact that it will introduce a correction that is based also on the step at time $t-1$, on the contrary the original idea produces a correction related only to the current point.
$$\Delta \omega_{pq}(t+1) = -\eta \frac{\partial E}{\partial w_{pq}}+\underbrace{\alpha \Delta w_{pq} (t)}_{\text{momentum }}$$

\begin{itemize}
	\item If $\alpha = 0$ we come back to the original formula $\Delta = f(w^T)$
	\item Otherwise $\Delta = f(w^T, w^{t-1})$
\end{itemize}
The momentum term introduces dependency with the previous step. Surely, a negative aspect is the fact that we need to set two parameters but on the other hand the momentum term allows us to use large values of $\eta$ avoiding the introduction of the oscillatory phenomena. The application of the momentum term can be seen in the following image.
\image{img/momentum_term}{Momentum term application.}{0.45}
Both trajectories uses $\eta=0.0476$ that is the best value of learning rate in the absence of momentum. On the left example there is not application of momentum $(\alpha=0)$, while on the right $\alpha=0.5$. The application of momentum, hence, brings a clear improvement in the convergence.

\paragraph*{The Problem of Local Minima.} One of the most important disadvantages is the fact that back-propagation cannot avoid the problem of local minima. For this reason the choice of initial weights is very important, if they are too large the nonlinearities tend to saturate since the beginning of the learning process. \\
A common heuristic for the choice of the initial weights is: 
$$w_{ij} \simeq 1/\sqrt{K_i} \qquad \text{Where } k_i \text{ is the number of units that feed unit }i \text{ (the "fan-in" of }i \text{)}$$
\image{img/problem_local_minima}{The problem of local minima.}{0.4}

\subsection{Theoretical / Practical Questions}
\begin{itemize}
	\item How many layers are needed for a given task?
	\item How many units per layer?
	\item To what extent does representation matter?
	\item What do we mean by generalization?
	\item What can we expect a network to generalize?
	\begin{itemize}
		\item \textbf{Generalization:} performance of the network on data not included in the training set. One of the major advantages of neural nets is their ability to generalize. This means that a trained net could classify data from the same class as the learning data that it has never seen before. In real world applications developers normally have only a small part of all possible patterns for the generation of a neural net. To reach the best generalization, the dataset should be split into three parts: training set, validation set and test set.\\
		The learning should be stopped in the minimum of the validation set error. At this point the net generalizes best. When learning is not stopped, overtraining occurs and the performance of the net on the whole data decreases, despite the fact that the error on the training data still gets smaller. After finishing the learning phase, the net should be finally checked with the third data set, the test set.
		\item Size of the training set: how large a training set should be for "good" generalization?
		\item Size of the network: too many weights in a network result in poor generalization.
	\end{itemize}
\end{itemize}

\subsection{Model evaluation}
When we talk about the model evaluation, it is important to understand specially in which way the goodness of the model can be evaluated. Surely the lower is the error generated by the model and the best is the fit created by the model. It is possible to differentiate between two different types of errors:
\begin{itemize}
	\item The \textbf{true error} (denoted $error_{\mathcal{D}}(h)$) of hypothesis $h$ with respect to target function $f$ and distribution $\mathcal{D}$, is the probability that $h$ will misclassify an instance drawn at random according to $\mathcal{D}$.
	$$error_{\mathcal{D}}(h) \equiv \operatorname{Pr}_{x \in \mathcal{D}} \left[f(x) \neq h(x)\right]$$ 
	In other words it measures the probability of committing error in the real life, but notice that for computing this quantity it is necessary to have knowledge of the probability distribution (which in general in unknown).
	
	\item The \textbf{sample error} (denoted $error_{s}(h)$) of hypothesis $h$ with respect to target function $f$ and data sample $S$ is:
	$$error_{S}(h) \equiv \frac{1}{n} \sum_{x \in S} \delta (f(x) , h(x))$$ 
	Where $n$ is the number of examples in $S$, and the quantity $\delta(f(x), h(x))$ is $1$ if $f(x) \neq h(x)$, and $0$ otherwise. It comes from the idea that since in general we don't know the probability distribution of our data we estimate it using the samples.
\end{itemize}
The true error is unknown (and will remain so forever$\dots$). In order to compute the sample error, it is possible to subdivide the dataset in a percentage for the training set and a percentage for the test set.
\image{img/train_test}{Training set vs Test set.}{0.4}
\paragraph{Cross-Validation} Cross-validation is a technique to evaluate predictive models by partitioning the original sample into a training set to train the model, and a test set to evaluate it. It avoids that test and training set affect the model evaluation. 
\image{img/cross_validation}{Cross validation example using
\textbf{leave-one-out} technique (size of the test fold = 1).}{0.6}
Cross validation is an useful method to evaluate the performance of a neural network. The drawback is the time required to run it.

\paragraph{Overfitting}
Even if the cross-validation technique can give us a good evaluation of the model, it is possible that this score is due to a particular problem called \textbf{overfitting}. In the following image, it is possible to see in the right image its phenomenon.
\image{img/overfitting}{Example of overfitting.}{0.65}
In the image \textbf{(a)}, a good fit to noisy data and the model uses less parameters capturing the general trend behavior. Instead in the image \textbf{(b)} overfitting of the same data is performed, the fit is perfect on the "training set", but it is likely to be poor on "test set" represented by the circle. Following the concept of the \textbf{Occam's razor} the simplest model is preferable.\\
The different steps in the model definition are reached through the separation of the starting set in:
\begin{itemize}
	\item \textbf{Training set:} for the execution of the learning algorithm.
	\item \textbf{Validation set:} it is used to stop the learning algorithm
	\item \textbf{Test set:} to evaluate the performance of the learning algorithm.
\end{itemize}
\image{img/early_stopping}{Early stopping.}{0.5}
Epoches of a learning algorithms represents the steps of machine learning algorithm. After the green line (overfitting line) during the next epochs the model start to learn noisy. Note also that the greater global optimum represents the overfit of the model, meaning that the error function is equal to zero since the model works well with training data points.
The stop of the learning algorithm is executed when the fit starts deteriorating.

\paragraph*{The size of the NN.} The size of a NN (the number of hidden neurons) affects both its functional capabilities and its generalization performance.
\begin{itemize}
	\item A \textbf{big network} leads to poor generalization performance. It has a lot of parameters and for this reason it's a too complex model.
	\item A \textbf{small network} could not be able to realize the desired input/output mapping and for this reason it won't be able to learn nothing.
\end{itemize}
However, it is hard to tell when the algorithm should stop because it is impossible to tell if the addiction of a further unit will dramatically decrease the error. This problem is known as \textbf{horizon effect}. A common strategy is called \textbf{growing}, it starts with one neuron and then it tries to train. It will add neurons until we will reach the right size.

\paragraph*{The Pruning Approach.} This approach is based on the idea of training an over-dimensioned network and then removing \textbf{redundant} nodes and connections. In pruning, indeed, we have a large number of neurons than the necessary and we will search to reduce the size. Pruning reduces the final complexity of the classifier and hence, it improves predictive accuracy by the reduction of overfitting.\\
The most important advantages of this technique are:
\begin{itemize}
	\item Arbitrarily complex decision regions.
	\item Faster training.
	\item Independence of the training algorithm.
\end{itemize}

The pruning approach has anyway some disadvantages: in a network with the perfect number of neurons backpropagation can fail, since the number of degree of freedom is limited. In general it is not a good idea to apply learning algorithms on a network with the exact number of neurons due to the limitation on the degrees of freedom. If we have more degrees of freedom we have more chance and different ways to reach the desired objective.\\
We can have two types of pruning algorithm:
\begin{itemize}
	\item \textbf{Online pruning:} during the training process it will decrease the size of the network.
	\item \textbf{Offline pruning:} it's preferable because it is possible to train the network in whatever way and the pruning is done only after the training process. 
\end{itemize}
Suppose (for simplicity) a network with one hidden layer and suppose that unit $h$ is to be removed. 
\image{img/pruning}{Example of the pruning approach.}{0.5}
As a consequence of this action we have to remove the incoming/outgoing connections and update the weights of the other connections such that the output remains the same. We focus on the input because if the input remains the same also the output remains unchanged.

This is equivalent to solving the system:
$$\underbrace{\sum_{j=1}^{n_{h}} w_{ij} y_{j}^{(\mu)}}_{\text{Before removing }h} = \underbrace{\sum_{j=1 \atop j\neq h}^{n_{h}} \left( w_{ij} + \delta_{ij} \right) y_{j}^{ (\mu)}}_{\text{After removing }h} \qquad i=1 \dots n_0 \text{ (\# output nodes)},~\mu=1 \dots P \text{ (network levels)}$$
Which is equivalent to the following linear system (in the unknown $\delta$'s):
$$\sum_{j \neq h} \delta_{ij} y_{j}^{(\mu)} = w_{ih} y_{h}^{(\mu)} \qquad i=1, \dots n_0, ~ \mu=1,\dots, P$$
We can observe that $\sum_{j \neq h} \delta_{ij} y_{j}^{(\mu)}$ is a linear system of equations and instead $w_{ih} y_{h}^{(\mu)}$ represents $b$.
\begin{equation*}
\begin{split}
\sum_{j=1}^{n_{h}} w_{ij} y_{j}^{(\mu)} &= \sum_{j=1 \atop j\neq h}^{n_{h}} \left( w_{ij} + \delta_{ij} \right) y_{j}^{ (\mu)} \\
& = \sum_{j=1 \atop j\neq h}^{n_{h}} w_{ij}y_{j}^{ (\mu)} + \sum_{j=1 \atop j\neq h}^{n_{h}}\delta_{ij}  y_{j}^{ (\mu)}
\end{split}
\end{equation*}
In order to satisfy this equation is necessary that $\sum_{j \neq h} \delta_{ij} y_{j}^{(\mu)} = w_{ih} y_{h}^{(\mu)}$ since the rest of the weights remains the same, the unique contribution that was deleted is the one provided by the neuron $h$.\\

In a more compact notation, it is possible to write the previous linear system as:
$$Ax = b \qquad \text{where } A \in \mathbb { R } ^ { P n _ { o } \times n _ { o } \left( n _ { h } - 1 \right) }$$
$A$ is a matrix that represents the weights of the nodes in the output layer fed by the node $h$, $x$ are the feature vectors and $b$ means the contribution to the network.\\
The \textbf{Least Squares solution} is:
$$min_{\bar{x}} \vert \vert A \bar{x} - b \vert \vert$$
A solution not always exists.\\
%TODO: Da spiegare meglio detecting excessive units
An important aspect is to detect which is the right neuron to prune. Residual reducing-methods start with an initial solution $x_0$ and produces some sequences of points $\{x_k\}$ so that the residuals are computed as:
$$\vert \vert Ax_0 - b \vert \vert  = r_0 \hspace{9.7em} $$
$$\vert \vert Ax_1 - b \vert \vert  = r_1 \hspace{9.7em}$$
$$\vdots \hspace{9em}$$
$$\vert\vert Ax_k - b \vert \vert = r_k \qquad \text{ where } r_k \leq r_{k-1}$$
The starting point is in $x_0 = 0 \rightarrow r_0=||b||$.
From this previous observation it is possible to say that excessive units can be detected so that $|| b ||$ is minimum.\\

The  algorithm instead of computing all the system considers only the initial contribution $||b||$ for each neuron, and then uses it as an estimate of the true contribution. Once the neuron with the smallest contribution is detected, the real linear system is computed and weights on the net are updated. It is an heuristic procedure that hopes that the initial contribution $||b||$ will be not to far from the real one.

\paragraph{Example} $b = w_{ih}y_n$ and $A= \sum\delta_{ij}y_j^{(\mu)}$

$$h1: \vert \vert A_1x_1 - b_1 \vert \vert  = 0.1 \hspace{9.7em} $$
$$h2: \vert \vert A_2x_1 - b_2 \vert \vert  = 1.6 \hspace{9.7em} $$
$$h3: \vert \vert A_3x_1 - b_3 \vert \vert  = 7.6 \hspace{9.7em} $$
From this example we can see that $h_1$ should be removed since the contribution of that neuron is the smallest one and it is very small. We can see that $ \vert \vert A_hx_h - b_h\vert \vert$ measures how different it the network after removing the neuron $h$.
The problem of this procedure is that we have to compute a lot of linear systems, one for each neuron.



\paragraph*{The Pruning Algorithm.}
The pruning algorithm follows these steps:
\begin{enumerate}
	\item Start with an over-sized trained network.
	\item Repeat
	\begin{enumerate}
		\item[2.1] Find the hidden unit $h$ for which $\vert \vert b \vert \vert$ is minimum.
		\item[2.2] Solve the corresponding system.
		\item[2.3] Remove unit $h$.
	\end{enumerate}
	Until $\text{Perf(pruned)} - \text{Perf(original)} < \varepsilon$
	\item Reject the last reduced network.
\end{enumerate}
We can formulate the goal al the pruning algorithm in this way:
$$\min_x \vert\vert Ax + b \vert\vert$$
\subsection{Deep Neural Networks}
The philosophy based on Deep Learning is to learn a feature hierarchy from the initial pixel image in order to obtain a classifier. Each layer extracts features from the output of previous layer and it trains all layers jointly.
\image{img/deep_learning}{Example of deep learning process.}{0.75}

\paragraph*{Shallow vs Deep Networks.} Shallow architectures are inefficient at representing deep functions. A shallow network, indeed, has less number of hidden layers. A shallow network with a large enough single hidden layer can fit any function (universal approximator), but on the other hand this produces a very large hidden layer and increase a lot the number of parameters. A deep network can fit functions better with less parameters than a shallow network, increasing the number of hidden layers but decreasing the number of required parameters. Deep networks try to simulate the brain behavior, in which the electronic signals traversal different layers.
\image{img/shallow_vs_deep}{Shallow vs Deep Networks.}{0.78}

Another important aspect to notice is the fact that there's an improvement of the performances with the presence of more data.
\image{img/performances}{Performances improves with more data.}{0.6}
The usage of deep networks is not a current idea, it is indeed an old idea, but it has been made possible only nowadays thanks to the fact that we have more data and we have more computing power (GPUs are really good at this).\\
The \textbf{image classification} consists in predicting a single label (or a distribution over labels as shown here to indicate our confidence) for a given image. Images are 3-dimensional arrays of integers from 0 to 255 of size Width x Height x 3. The 3 represents the three color channels Red, Green and Blue.
\image{img/image_classification}{Image classification example.}{0.5}
In image classification, the most important challenges are:
\image{img/challenges}{Challenges in image classification.}{0.95}
For this reason, in order to face these challenges, what we need is a data-driven approach in which we have thousands of categories and hundreds of thousand of images for each category.\\
As it is possible to understand, there's a difference between \textbf{traditional approach} and \textbf{deep learning}, indeed, in the first one we extract meaningful features from images through a manual process, in the second case instead, everything is automatic and there's a bunch of layers in which the final ones are useful for classification.

\paragraph*{Convolutional Neural Networks (CNNs).} Convolutional Neural Networks are used to solve OCR problems. A neural network can be defined as a \textbf{fully-connected network} or a \textbf{local-connected network}. A local-connected layer is much more specialized, and efficient, than a fully-connected layer. In a fully-connected layer each neuron is connected to every neuron in the previous layer, and each connection has it's own weight. This is a totally general purpose connection pattern and makes no assumptions about the features in the data. It's also very expensive in terms of memory (weights) and computations (connections).\\
In contrast, in a local-connected layer, each neuron is only connected to a few nearby neurons in the previous layer, and the same set of weights (and local connection layout) is used for every neuron. This connection pattern only makes sense for cases where the data can be interpreted as spatial with the features to be extracted being spatially local and equally likely to occur at any input position. The typical use case for local-connected layers is for image data where, as required, the features are local (e.g. a "nose" consists of a set of nearby pixels, not spread all across the image). The fewer number of connections and weights make local-connected layers relatively cheap in terms of memory and computing power needed.
\image{img/fully_local}{Fully vs local-connected networks.}{0.75}
Normally, several filters are packed together and learnt automaticaly during training.
\image{img/trainableFilters}{Using Several Trainable Filters.}{0.75}

\textbf{Max pooling} is a way to simplify the network architecture, by downsampling the number of neurons resulting from the filtering operations. In other words, the image is partitioned in small squares and for each square the maximum value pixel is taken.
\image{img/max_pooling}{Max Pooling.}{0.6}
As we can see in the next image, a deep neural network is a combination of feature extraction and classification processes. In input we have an image, then there are some initial layers locally-connected that extract features from the image. In the final part there's the classification part and it is fully-connected.
\image{img/extraction_and_classification}{Combining Feature Extraction and Classification.}{0.7}

\paragraph*{AlexNet Architecture.} AlexNet presents 8 layers total that follows the following schema:
\imageb{img/alexNetLayers}{0.18}
Going more in deep inside these layers, we can notice that:
\begin{itemize}
	\item \textbf{1st layer:} 96 kernels $(11 \times 11 \times 3)$.
	\item \textbf{2nd layer:} 256 kernels $(5 \times 5 \times 48)$.
	\item \textbf{3rd layer:} 384 kernels $(3 \times 3 \times 256)$.
	\item \textbf{4th layer:} 384 kernels $(3 \times 3 \times 192)$.
	\item \textbf{5th layer:} 256 kernels $(3 \times 3 \times 192)$.
	\item \textbf{6th layer:} fully connected layer with 4096 neurons.
	\item \textbf{7th layer:} fully connected layer with 4096 neurons.
	\item \textbf{8th layer:} 1000-way \textbf{SoftMax} layer.
\end{itemize}
The graphical representation of the AlexNet can be seen in the following image. There are two independent GPUs that runs in parallel and it can be noticed that there are connections between the two GPUs. The last \textit{SoftMax} layer gives as output:
$$y_i = \frac{e^{\text{net}_i}}{\sum_j e^{\text{net}_i}}$$
\image{img/alexNet}{AlexNet architecture.}{0.8}

\paragraph*{ReLU's.} \textbf{ReLU} is an acronym that stands for \textit{Rectified Linear Units}, they are used to solve the problem that sigmoid activation takes only values in $(0,1)$. Propagating the gradient back to the initial layers, it tends to become $0$ (\textit{vanishing gradient problem}). From a practical perspective, this slows down the training procedure of the initial layers of the network. Indeed, with sigmoid the gradient will be close to 0 and the algorithm won't make progresses, in order to speed up the progress of learning, ReLU is used.
\image{img/relu}{Comparison between sigmoid and ReLU functions.}{0.6}
It is possible to notice also that ReLU reaches the same results but much faster than sigmoid function. As it can be seen in the following image, the solid line represents the convergence of a 4 layer CNN with ReLUs, the dashed line instead represents an equivalent network with $tanh$ neurons. It can be noticed that the CNN with ReLUs converges six times faster.
\image{img/relu2}{Convergence of ReLU neurons.}{0.4}

%TODO: specificare meglio.
\paragraph*{Mini-batch Stochastic Gradient Descent.} We extract a random sample and we compute the gradient on it. Each time we execute it we have a different result. At each loop these steps are done:
\begin{enumerate}
	\item Sample a batch of data (the dimension is an hyper parameter to choose).
	\item Forward prop it through the graph, get loss.
	\item Backprop to calculate the gradients.
	\item Update the parameters using the gradient.
\end{enumerate} 

The easiest and most common method to reduce the overfitting on image data is to artificially enlarge the dataset using label-preserving transformations.\\
AlexNet uses two forms of this \textbf{data augmentation}:
\begin{itemize}
	\item The first form consists of generating image translations and horizontal reflections (translations, scaling or rotations of the images).
	\item The second form consists of altering the intensities of the RGB channels in training images (introduce random noise inside the images). 
\end{itemize}

\paragraph*{Dropout.} Set to 0 the output of each hidden neuron with probability $0.5$. The neurons which are \textit{"dropped out"} in this way do not contribute to the forward pass and do not participate in back-propagation. So every time an input is presented, the neural network samples a different architecture, but all these architectures share weights.\\
Dropout reduces complex co-adaptations of neurons, since a neuron cannot rely on the presence of particular other neurons.
\image{img/dropout}{Dropout results}{0.55}
